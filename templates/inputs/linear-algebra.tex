\usepackage{etoolbox}
\usepackage{xparse}
\usepackage{amsthm}

\newcommand{\mymat}[1]{%
  \begin{bmatrix}
    #1
  \end{bmatrix}
}% end \mymat

\newcommand{\colvec}[1]{%
  \renewcommand*{\do}[1]{##1 \\}
  \begin{bmatrix}[r]
    \docsvlist{#1}
  \end{bmatrix}
}% end \colvec

\NewDocumentCommand{\colvecn}{ O{x} O{3} }{%
  \begin{bmatrix}
   \iteratevecn{#1}{#2}
  \end{bmatrix}
}% end \colvecn

\newcounter{vecncounter}
\newbool{vecnlength}
\newcommand{\iteratevecn}[2]{
  \defcounter{vecncounter}{0}
  \unlessboolexpr{bool {vecnlength}}{
    \stepcounter{vecncounter}
    \ifnumcomp{\thevecncounter}{>}{#2}
      {\setbool{vecnlength}{true}}
      {
        \ifnumcomp{\thevecncounter}{=}{#2}
          {{#1}_{\thevecncounter}}
          {{#1}_{\thevecncounter} \\ }
      }
  }% end \unlessboolexpr
}% end \iteratevecn


\newcommand{\colvecton}[1][x]{%
  \begin{bmatrix}[c]
    {#1}_1 \\ \vdots \\ {#1}_n
  \end{bmatrix}
}% end \colvecton


% \newcommand{\colmat}[1]{%
%   \renewcommand*{\do}[1]{ | \\ ##1 \\ |}
%   \begin{bmatrix}[r]
%     \docsvlist{#1}
%   \end{bmatrix}
% }% end \colmat

\theoremstyle{theorem}
\newtheorem*{theorem}{Theorem}

\theoremstyle{definition}
\newtheorem*{definition}{Definition}

\theoremstyle{remark}
\newtheorem*{note}{Note}

\newcommand\R{\mathbb{R}}
\newcommand\Rn{\mathbb{R}^{n}}
\newcommand\Rm{\mathbb{R}^{m}}
\newcommand\linearsystem[1]{\sysdelim[]\systeme{#1}}

\newcommand\for{\text{for}\;}
\newcommand\where{\text{where}\;}
\newcommand\andd{\text{and}\;}

\DeclareMathOperator{\rref}{rref}
\DeclareMathOperator{\proj}{proj}
\DeclareMathOperator{\reff}{ref}
\DeclareMathOperator{\spann}{span}
\DeclareMathOperator{\image}{image}
\DeclareMathOperator{\kerr}{ker}
\DeclareMathOperator{\rank}{rank}

% For augmented matrices
\makeatletter
\renewcommand*\env@matrix[1][*\c@MaxMatrixCols c]{%
  \hskip -\arraycolsep
  \let\@ifnextchar\new@ifnextchar
  \array{#1}}
\makeatother
